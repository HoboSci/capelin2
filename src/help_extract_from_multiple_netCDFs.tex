\documentclass[]{article}
\usepackage{lmodern}
\usepackage{amssymb,amsmath}
\usepackage{ifxetex,ifluatex}
\usepackage{fixltx2e} % provides \textsubscript
\ifnum 0\ifxetex 1\fi\ifluatex 1\fi=0 % if pdftex
  \usepackage[T1]{fontenc}
  \usepackage[utf8]{inputenc}
\else % if luatex or xelatex
  \ifxetex
    \usepackage{mathspec}
  \else
    \usepackage{fontspec}
  \fi
  \defaultfontfeatures{Ligatures=TeX,Scale=MatchLowercase}
\fi
% use upquote if available, for straight quotes in verbatim environments
\IfFileExists{upquote.sty}{\usepackage{upquote}}{}
% use microtype if available
\IfFileExists{microtype.sty}{%
\usepackage{microtype}
\UseMicrotypeSet[protrusion]{basicmath} % disable protrusion for tt fonts
}{}
\usepackage[margin=1in]{geometry}
\usepackage{hyperref}
\hypersetup{unicode=true,
            pdftitle={help\_extract\_from\_multiple\_netCDFs},
            pdfauthor={Samantha Andrews},
            pdfborder={0 0 0},
            breaklinks=true}
\urlstyle{same}  % don't use monospace font for urls
\usepackage{color}
\usepackage{fancyvrb}
\newcommand{\VerbBar}{|}
\newcommand{\VERB}{\Verb[commandchars=\\\{\}]}
\DefineVerbatimEnvironment{Highlighting}{Verbatim}{commandchars=\\\{\}}
% Add ',fontsize=\small' for more characters per line
\usepackage{framed}
\definecolor{shadecolor}{RGB}{248,248,248}
\newenvironment{Shaded}{\begin{snugshade}}{\end{snugshade}}
\newcommand{\KeywordTok}[1]{\textcolor[rgb]{0.13,0.29,0.53}{\textbf{#1}}}
\newcommand{\DataTypeTok}[1]{\textcolor[rgb]{0.13,0.29,0.53}{#1}}
\newcommand{\DecValTok}[1]{\textcolor[rgb]{0.00,0.00,0.81}{#1}}
\newcommand{\BaseNTok}[1]{\textcolor[rgb]{0.00,0.00,0.81}{#1}}
\newcommand{\FloatTok}[1]{\textcolor[rgb]{0.00,0.00,0.81}{#1}}
\newcommand{\ConstantTok}[1]{\textcolor[rgb]{0.00,0.00,0.00}{#1}}
\newcommand{\CharTok}[1]{\textcolor[rgb]{0.31,0.60,0.02}{#1}}
\newcommand{\SpecialCharTok}[1]{\textcolor[rgb]{0.00,0.00,0.00}{#1}}
\newcommand{\StringTok}[1]{\textcolor[rgb]{0.31,0.60,0.02}{#1}}
\newcommand{\VerbatimStringTok}[1]{\textcolor[rgb]{0.31,0.60,0.02}{#1}}
\newcommand{\SpecialStringTok}[1]{\textcolor[rgb]{0.31,0.60,0.02}{#1}}
\newcommand{\ImportTok}[1]{#1}
\newcommand{\CommentTok}[1]{\textcolor[rgb]{0.56,0.35,0.01}{\textit{#1}}}
\newcommand{\DocumentationTok}[1]{\textcolor[rgb]{0.56,0.35,0.01}{\textbf{\textit{#1}}}}
\newcommand{\AnnotationTok}[1]{\textcolor[rgb]{0.56,0.35,0.01}{\textbf{\textit{#1}}}}
\newcommand{\CommentVarTok}[1]{\textcolor[rgb]{0.56,0.35,0.01}{\textbf{\textit{#1}}}}
\newcommand{\OtherTok}[1]{\textcolor[rgb]{0.56,0.35,0.01}{#1}}
\newcommand{\FunctionTok}[1]{\textcolor[rgb]{0.00,0.00,0.00}{#1}}
\newcommand{\VariableTok}[1]{\textcolor[rgb]{0.00,0.00,0.00}{#1}}
\newcommand{\ControlFlowTok}[1]{\textcolor[rgb]{0.13,0.29,0.53}{\textbf{#1}}}
\newcommand{\OperatorTok}[1]{\textcolor[rgb]{0.81,0.36,0.00}{\textbf{#1}}}
\newcommand{\BuiltInTok}[1]{#1}
\newcommand{\ExtensionTok}[1]{#1}
\newcommand{\PreprocessorTok}[1]{\textcolor[rgb]{0.56,0.35,0.01}{\textit{#1}}}
\newcommand{\AttributeTok}[1]{\textcolor[rgb]{0.77,0.63,0.00}{#1}}
\newcommand{\RegionMarkerTok}[1]{#1}
\newcommand{\InformationTok}[1]{\textcolor[rgb]{0.56,0.35,0.01}{\textbf{\textit{#1}}}}
\newcommand{\WarningTok}[1]{\textcolor[rgb]{0.56,0.35,0.01}{\textbf{\textit{#1}}}}
\newcommand{\AlertTok}[1]{\textcolor[rgb]{0.94,0.16,0.16}{#1}}
\newcommand{\ErrorTok}[1]{\textcolor[rgb]{0.64,0.00,0.00}{\textbf{#1}}}
\newcommand{\NormalTok}[1]{#1}
\usepackage{graphicx,grffile}
\makeatletter
\def\maxwidth{\ifdim\Gin@nat@width>\linewidth\linewidth\else\Gin@nat@width\fi}
\def\maxheight{\ifdim\Gin@nat@height>\textheight\textheight\else\Gin@nat@height\fi}
\makeatother
% Scale images if necessary, so that they will not overflow the page
% margins by default, and it is still possible to overwrite the defaults
% using explicit options in \includegraphics[width, height, ...]{}
\setkeys{Gin}{width=\maxwidth,height=\maxheight,keepaspectratio}
\IfFileExists{parskip.sty}{%
\usepackage{parskip}
}{% else
\setlength{\parindent}{0pt}
\setlength{\parskip}{6pt plus 2pt minus 1pt}
}
\setlength{\emergencystretch}{3em}  % prevent overfull lines
\providecommand{\tightlist}{%
  \setlength{\itemsep}{0pt}\setlength{\parskip}{0pt}}
\setcounter{secnumdepth}{0}
% Redefines (sub)paragraphs to behave more like sections
\ifx\paragraph\undefined\else
\let\oldparagraph\paragraph
\renewcommand{\paragraph}[1]{\oldparagraph{#1}\mbox{}}
\fi
\ifx\subparagraph\undefined\else
\let\oldsubparagraph\subparagraph
\renewcommand{\subparagraph}[1]{\oldsubparagraph{#1}\mbox{}}
\fi

%%% Use protect on footnotes to avoid problems with footnotes in titles
\let\rmarkdownfootnote\footnote%
\def\footnote{\protect\rmarkdownfootnote}

%%% Change title format to be more compact
\usepackage{titling}

% Create subtitle command for use in maketitle
\newcommand{\subtitle}[1]{
  \posttitle{
    \begin{center}\large#1\end{center}
    }
}

\setlength{\droptitle}{-2em}

  \title{help\_extract\_from\_multiple\_netCDFs}
    \pretitle{\vspace{\droptitle}\centering\huge}
  \posttitle{\par}
    \author{Samantha Andrews}
    \preauthor{\centering\large\emph}
  \postauthor{\par}
    \date{}
    \predate{}\postdate{}
  

\begin{document}
\maketitle

A note to anyone who might happen to stumble across this\ldots{} I am a
beginner in R and have had no exposure to similar languages. I don't
know what I'm doing. The code herein is unlikely to be elegant and there
area probably more efficient ways of running the code.

Built with

\begin{Shaded}
\begin{Highlighting}[]
\KeywordTok{getRversion}\NormalTok{()}
\end{Highlighting}
\end{Shaded}

\begin{verbatim}
## [1] '3.5.1'
\end{verbatim}

\section{Package dependencies}\label{package-dependencies}

You can install and load them using the following code which uses a
function called
\href{https://gist.github.com/stevenworthington/3178163}{ipak}. Note
this function checks to see if the packages are installed first. If they
are not installed, it will do so before loading

\begin{Shaded}
\begin{Highlighting}[]
\NormalTok{packages <-}\StringTok{ }\KeywordTok{c}\NormalTok{(}\StringTok{"ncdf4"}\NormalTok{, }\StringTok{"sp"}\NormalTok{, }\StringTok{"raster"}\NormalTok{) }
\KeywordTok{source}\NormalTok{(}\StringTok{"../src/ipak.R"}\NormalTok{)}
\KeywordTok{ipak}\NormalTok{(packages)}
\end{Highlighting}
\end{Shaded}

\begin{verbatim}
## Loading required package: ncdf4
\end{verbatim}

\begin{verbatim}
## Loading required package: sp
\end{verbatim}

\begin{verbatim}
## Loading required package: raster
\end{verbatim}

\begin{verbatim}
##  ncdf4     sp raster 
##   TRUE   TRUE   TRUE
\end{verbatim}

\section{The short story}\label{the-short-story}

I have multiple netCDFs that need to extract values from to points. The
problem is, I need R to decide which netCDFs (and which layers) to
perform the extraction on.

\section{The details}\label{the-details}

I have around 1500 NetCDFs. Each netCDF has 4 dimensions
(\texttt{time\ -\ T}, \texttt{longitude\ -\ X}, \texttt{latitude\ -\ Y},
\texttt{depth\ -\ Z}) and represents a single oceanographic variable for
a particular month in a particular year. I have given each netCDF the
following naming convention:

\texttt{yyyy\_mm\_variablename.nc} (where variable name can vary in
length)

An example of the \texttt{netCDF} is available
\href{https://drive.google.com/file/d/1y2XU5RzUBRLgTXkql667a5o_PVk5LMTP/view?usp=sharing}{here}
. To use the netCDFs in R, I have made a loop to create
\texttt{raster\ bricks} and project them in an equal-area grid:

\begin{Shaded}
\begin{Highlighting}[]
\NormalTok{aea <-}\StringTok{ }\KeywordTok{raster}\NormalTok{(}\StringTok{"../output/env/aea.tif"}\NormalTok{)}
\NormalTok{netcdf_list <-}\StringTok{ }\KeywordTok{list.files}\NormalTok{(}\StringTok{"../data/env/netcdf"}\NormalTok{, }\DataTypeTok{pattern =} \StringTok{'*.nc'}\NormalTok{, }\DataTypeTok{full.names =} \OtherTok{TRUE}\NormalTok{) }\CommentTok{#true means the full path is included}
\NormalTok{no_netcdf <-}\StringTok{ }\KeywordTok{length}\NormalTok{(netcdf_list) }\CommentTok{#for the loop - need to know how many files to cycle through}
\NormalTok{netcdf_name <-}\StringTok{ }\KeywordTok{list.files}\NormalTok{(}\StringTok{"../data/env/netcdf"}\NormalTok{, }\DataTypeTok{pattern =} \StringTok{'*.nc'}\NormalTok{, }\DataTypeTok{full.names =} \OtherTok{FALSE}\NormalTok{) }\CommentTok{#false means the path is not included}
\KeywordTok{library}\NormalTok{ (ncdf4)}
\ControlFlowTok{for}\NormalTok{ (i }\ControlFlowTok{in} \DecValTok{1}\OperatorTok{:}\NormalTok{no_netcdf) \{}
  \KeywordTok{print}\NormalTok{(netcdf_name[i]) }\CommentTok{#this just prings the name of the netCDF R is working on}
\NormalTok{  temp_brick <-}\StringTok{ }\KeywordTok{brick}\NormalTok{(netcdf_list[i], }\DataTypeTok{lvar =} \DecValTok{4}\NormalTok{)}
\NormalTok{  temp_brick <-}\StringTok{ }\KeywordTok{projectRaster}\NormalTok{(temp_brick, aea) }\CommentTok{#aea is an existing raster in the projection I want that I created (environmental_data_preperation)}
  \KeywordTok{assign}\NormalTok{(netcdf_name[i], temp_brick) }\CommentTok{#this asigns the netCDF name to the raster brick}
\NormalTok{\}}
\end{Highlighting}
\end{Shaded}

\begin{verbatim}
## [1] "1998_01_mlp.nc"
## [1] "1998_01_salinity.nc"
## [1] "1998_01_ssh.nc"
## [1] "1998_01_temp.nc"
## [1] "1998_02_mlp.nc"
## [1] "1998_02_salinity.nc"
## [1] "1998_02_ssh.nc"
## [1] "1998_02_temp.nc"
## [1] "1998_03_mlp.nc"
## [1] "1998_03_salinity.nc"
## [1] "1998_03_ssh.nc"
## [1] "1998_03_temp.nc"
## [1] "1998_04_mlp.nc"
## [1] "1998_04_salinity.nc"
## [1] "1998_04_ssh.nc"
## [1] "1998_04_temp.nc"
## [1] "1998_05_mlp.nc"
## [1] "1998_05_salinity.nc"
## [1] "1998_05_ssh.nc"
## [1] "1998_05_temp.nc"
## [1] "1998_06_mlp.nc"
## [1] "1998_06_salinity.nc"
## [1] "1998_06_ssh.nc"
## [1] "1998_06_temp.nc"
## [1] "1998_07_mlp.nc"
## [1] "1998_07_salinity.nc"
## [1] "1998_07_ssh.nc"
## [1] "1998_07_temp.nc"
## [1] "1998_08_mlp.nc"
## [1] "1998_08_salinity.nc"
## [1] "1998_08_ssh.nc"
## [1] "1998_08_temp.nc"
## [1] "1998_09_mlp.nc"
## [1] "1998_09_salinity.nc"
## [1] "1998_09_ssh.nc"
## [1] "1998_09_temp.nc"
## [1] "1998_10_mlp.nc"
## [1] "1998_10_salinity.nc"
## [1] "1998_10_ssh.nc"
## [1] "1998_10_temp.nc"
## [1] "1998_11_mlp.nc"
## [1] "1998_11_salinity.nc"
## [1] "1998_11_ssh.nc"
## [1] "1998_11_temp.nc"
## [1] "1998_12_mlp.nc"
## [1] "1998_12_salinity.nc"
## [1] "1998_12_ssh.nc"
## [1] "1998_12_temp.nc"
\end{verbatim}

Note that this may not be the most computationally efficient way to do
this as I will be creating a lot of raster bricks that will be stored to
memory\ldots{}

I also had a dataset (.csv) containing fish observation (point) data.
This dataset (let's call it \texttt{occurrence}) contains the following
columns:

\texttt{occurrence\$id} (a unique observation ID)
\texttt{occurrence\$decimalLongitude} (Longitude in decimal degree)
\texttt{occurrence\$decimalLatitude} (Latitude in decimal degree)
\texttt{occurrence\$dLongitudeM} (Longitude in meters)
\texttt{occurrence\$LatitudeM} (Latitude in meters)
\texttt{occurrence\$year} (year of observation)
\texttt{occurrence\$month} (month of observation)
\texttt{occurrence\$depthlayerno\ (depth\ of\ observation\ -\ the\ value\ corresponds\ to\ the\ depth\ layers\ in\ the\ raster\ bricks\ e.g.\ if}occurrence\$depthlayerno\texttt{==}20\texttt{,\ then\ the\ observation\ is\ from\ the\ raster\ brick\ layer}{[}{[}20{]}{]}`)

I then have some empty columns that need to be populated with data from
the raster bricks, each representing an oceanographic variable and depth
layer the data needs to come from. An example is:

\texttt{occurrence\$salinity\_1} (extract value to point from the
salinity raster brick layer \texttt{{[}{[}1{]}{]}})
\texttt{occurrence\$salinity\_depth} (extract value to point from the
salinity raster brick layer
\texttt{{[}{[}occurrence\$depthlayerno{]}{]}})

\begin{Shaded}
\begin{Highlighting}[]
\NormalTok{occurrence <-}\StringTok{ }\KeywordTok{read.csv}\NormalTok{(}\StringTok{"../data/bio/occurrence_help.csv"}\NormalTok{, }\DataTypeTok{header =} \OtherTok{TRUE}\NormalTok{)}
\end{Highlighting}
\end{Shaded}

Most of the raster bricks contain 50 layers - some only have 1 layer.
Furthermore not all observations have a depthlayerno value
(\texttt{occurrence\$depthlayerno\ ==\ NA})

A cut-down version of this .csv is available
\href{https://drive.google.com/file/d/13ZTZC48o2i0ZktobfukHjaTMh7Qbrfei/view?usp=sharing}{here}.

The \texttt{dput} for occurrence is as follows:

\begin{Shaded}
\begin{Highlighting}[]
\KeywordTok{dput}\NormalTok{(occurrence)}
\end{Highlighting}
\end{Shaded}

\begin{verbatim}
## structure(list(year = c(1998L, 1998L, 1998L, 1998L, 1998L, 1998L, 
## 1998L, 1998L, 1998L, 1998L, 1998L, 1998L, 1998L, 1998L, 1998L
## ), month = c(10L, 10L, 10L, 10L, 11L, 11L, 12L, 12L, 12L, 4L, 
## 4L, 4L, 4L, 4L, 5L), id = c(386669852L, 386669881L, 386669904L, 
## 386411062L, 386674153L, 386415830L, 386417305L, 386417335L, 386417588L, 
## 386403612L, 386402081L, 386402062L, 386402787L, 386401885L, 386405989L
## ), decimalLatitude = c(53.95, 53.63, 53.72, 45.19, 50.26, 48.42, 
## 48.96, 48.95, 47.7, 46.87, 46.38, 46.46, 47.4, 47.12, 45.6), 
##     decimalLongitude = c(-54.91, -55.51, -55.73, -52.85, -55.49, 
##     -49.52, -52.78, -52.81, -52.93, -57.86, -54.94, -54.82, -56.37, 
##     -54.61, -51.68), LongitudeM = c(331170.3733, 294592.1288, 
##     279536.5639, 568120.6623, 321164.228, 775577.4971, 529039.8088, 
##     526968.5028, 532747.635, 164445.6561, 392590.9816, 401204.4947, 
##     275695.0622, 411603.073, 655083.2443), LatitudeM = c(1549820.727, 
##     1511221.289, 1520353.142, 594666.88, 1135604.497, 981130.7983, 
##     1008824.847, 1007486.149, 868898.3383, 751816.2887, 709993.6159, 
##     719487.3038, 815139.5931, 793297.9028, 650032.9389), depthlayerno = c(30L, 
##     28L, 26L, 24L, 35L, 35L, 33L, 32L, 25L, 33L, 31L, 32L, 35L, 
##     31L, 23L), salinty_1 = c(NA, NA, NA, NA, NA, NA, NA, NA, 
##     NA, NA, NA, NA, NA, NA, NA), salinity_depth = c(NA, NA, NA, 
##     NA, NA, NA, NA, NA, NA, NA, NA, NA, NA, NA, NA), temp_1 = c(NA, 
##     NA, NA, NA, NA, NA, NA, NA, NA, NA, NA, NA, NA, NA, NA), 
##     temp_depth = c(NA, NA, NA, NA, NA, NA, NA, NA, NA, NA, NA, 
##     NA, NA, NA, NA)), class = "data.frame", row.names = c(NA, 
## -15L))
\end{verbatim}

And \texttt{str(occurrence)} is

\begin{Shaded}
\begin{Highlighting}[]
\KeywordTok{str}\NormalTok{(occurrence)}
\end{Highlighting}
\end{Shaded}

\begin{verbatim}
## 'data.frame':    15 obs. of  12 variables:
##  $ year            : int  1998 1998 1998 1998 1998 1998 1998 1998 1998 1998 ...
##  $ month           : int  10 10 10 10 11 11 12 12 12 4 ...
##  $ id              : int  386669852 386669881 386669904 386411062 386674153 386415830 386417305 386417335 386417588 386403612 ...
##  $ decimalLatitude : num  54 53.6 53.7 45.2 50.3 ...
##  $ decimalLongitude: num  -54.9 -55.5 -55.7 -52.9 -55.5 ...
##  $ LongitudeM      : num  331170 294592 279537 568121 321164 ...
##  $ LatitudeM       : num  1549821 1511221 1520353 594667 1135604 ...
##  $ depthlayerno    : int  30 28 26 24 35 35 33 32 25 33 ...
##  $ salinty_1       : logi  NA NA NA NA NA NA ...
##  $ salinity_depth  : logi  NA NA NA NA NA NA ...
##  $ temp_1          : logi  NA NA NA NA NA NA ...
##  $ temp_depth      : logi  NA NA NA NA NA NA ...
\end{verbatim}

I have converted the occurrence dataset into a
\texttt{SpatialPointsDataFrame}, projected it into an equal area grid
(to match the raster bricks), and them split the data into year-months
(I thought this might be an easier way to tackle the next stage):

\begin{Shaded}
\begin{Highlighting}[]
\KeywordTok{library}\NormalTok{(sp)}
\NormalTok{xy <-}\StringTok{ }\NormalTok{occurrence[ , }\KeywordTok{c}\NormalTok{(}\StringTok{"decimalLongitude"}\NormalTok{, }\StringTok{"decimalLatitude"}\NormalTok{)]}
\NormalTok{species_obs_date <-}\StringTok{ }\KeywordTok{SpatialPointsDataFrame}\NormalTok{(}\DataTypeTok{coords =}\NormalTok{ xy, }\DataTypeTok{data =}\NormalTok{ occurrence, }\DataTypeTok{proj4string =} \KeywordTok{CRS}\NormalTok{(}\StringTok{"+proj=longlat +datum=WGS84 +ellps=WGS84 +towgs84=0,0,0"}\NormalTok{)) }\CommentTok{#convert into a spatialpoints dataframe to do the extraction}
\NormalTok{species_obs_date <-}\StringTok{ }\KeywordTok{spTransform}\NormalTok{(species_obs_date, }\DataTypeTok{CRS =} \StringTok{"+proj=aea +lat_1=50 +lat_2=70 +lat_0=40 +lon_0=-60 +x_0=0 +y_0=0 +ellps=GRS80 +datum=NAD83 +units=m +no_defs"}\NormalTok{) }\CommentTok{#need to reproject.... note -60 used to be -91. Changed to 'straighten up'}
\NormalTok{split_obs <-}\StringTok{ }\KeywordTok{split}\NormalTok{(species_obs_date, }\KeywordTok{paste}\NormalTok{(species_obs_date}\OperatorTok{$}\NormalTok{year, species_obs_date}\OperatorTok{$}\NormalTok{month))}
\end{Highlighting}
\end{Shaded}

\section{The problem}\label{the-problem}

What I would like to happen is for each each observation (row) in
\texttt{occurrences}, for each oceanographic variable (e.g.~using
salinity) look up \texttt{occurrence\$year} and
\texttt{occurrence\$month} and locate the appropriate raster brick, and
extract values to points from layer \texttt{{[}{[}1{]}{]}} and add the
result to \texttt{occurrence\$salinity\_1}, then look up
\texttt{occurrence\$depthlayerno} and extract values to points from
layer\texttt{{[}{[}depthlayerno{]}{]}} and add the result to
\texttt{occurrence\$salinity\_depth}.

Honestly, I have no idea where to begin. I found a rather
\href{http://nathanlane.info/tutorial/2016/01/02/gisfunctional.html}{neat
blog from Nathan Lane} in which he uses functions to extract values to
polygons from multiple raster bricks (but without selecting particular
raster bricks to extract the data from). Unfortunately adapting it to my
needs is way beyond my current skillset.

Any help you can offer would be gratefully received!


\end{document}
